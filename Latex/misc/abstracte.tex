\section*{Abstract}	% section* heisst die section taucht nicht in der toc auf
\begin{spacing}{1.2}

In this master thesis, a new \ac{MitM} attack on a voice connection in the \ac{GSM} network is being developed. The attack exploits the lack of integrity protection on the radio interface to manipulate a victim's outgoing call and redirect it to a mobile phone designated by the attacker. In the theoretical part of the thesis, the attack and the security flaws it is based on, as well as relations to current attacks are established. In the practical part, the call setup manipulation is verified by a \ac{MitM} on a virtual radio interface. The encryption methods used in \ac{GSM} are stream ciphers. This means the data stream is secured by combining it with a key stream using the exclusive-or (\acs{XOR}) operation. It is shown that a cipher stream generated by a stream cipher can be arbitrarily manipulated if the plaintext is known. Since \ac{GSM} does not specify a procedure for integrity protection, the called phone number in the outgoing call can be replaced by the attacker. The attacker can redirect the call to a mobile entity under control and thus receives the voice traffic from the calling victim. By linking this voice traffic with a new call to the original called number, the attacker creates a \ac{MitM} in between the two communication partners. Because incoming and outgoing calls are valid \ac{GSM} calls, the network and mobile devices take care of the encryption and encoding of the voice data by themselves. This means the attacker has access to the unencrypted communication and can record or manipulate the call as they wish. Since the encryption does not have to be broken, the attack is working even with strong encryption algorithms. Furthermore, the use of the more recent 3G \ac{AKA} instead of the outdated 2G \ac{AKA} provides no protection from the attack. It brings mutual authentication to \ac{GSM} networks, but does not ensure integrity between \ac{MS} and \ac{BTS}. The attack requires partial known plaintext of the called mobile number, which can be assumed in \ac{GSM}. Mobile phone numbers usually have known parts, like the \ac{NDC} or the \ac{CC}. The installation of a \ac{MitM} on the physical level of the real radio frequency interface is not investigated in this work. However, the physical level of the radio interface is virtualized and the \ac{MitM} is inserted into it for the verification of the attack in the practical part. The implementations of the virtual physical interface as well as the \ac{MitM} are carried out within the framework of the Osmocom project.
\end{spacing}
\vfill
\acresetall