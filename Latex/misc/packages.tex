%%% Im folgenden Abschnitt können packages eingebunden werden. Packages erweitern Latex um Zusätzliche Befehle und oder passen an wie etwas dargestellt wird. %%%
\usepackage[T1]{fontenc}% Fontkodierung auf das neuere T1 umschalten, behebt umlautfehler beim kopieren aus generierter pdf
\usepackage{lmodern} % needed for scalable fonts
\usepackage[utf8]{inputenc} % Der Latex Code wird in utf8 kodiert. Das ist eine verbreitete Norm die praktisch alle Buchstaben und Sonderzeichen enthält.
\usepackage{microtype}
\usepackage{a4} % Setzt das Format des Dokuments auf DinA4
\usepackage{hyphenat} % Verwaltet Wörter, die nicht richtig getrennt werden und einen Zeilenüberlauf generieren würden.
\usepackage[hyphens]{url} % erlaubt mehr Umbrüche in urls, sehr wichtig für style des Literaturverzeichnisses
\usepackage[ngerman]{babel} % passt bestimmte Ausgaben wie z.B. Datum an die jeweilige Sprache an.
%\usepackage{parskip} % keine automatischer abstand nach paragrafen, use option parskip for KOMA Script classes
%\usepackage{umlaute} % umlaute werden richtig gerendert
\usepackage{adjustbox} % resizing stuff
\usepackage{graphicx} % ermöglicht einbinden von Bildern
\usepackage{setspace} % damit kann man den Zeilenabstand ändern (einfach, eineinhalbfach, doppelt)
%\usepackage{titlesec} % ermöglicht anpassen vom style der chapters, section und subsections
\usepackage{tabularx} % braucht man um Tabellen zu erstellen
\usepackage{wrapfig} % Formatiert Fießtext um Inline Bilder
\usepackage{floatflt} % Ermöglicht Inline Bilder und Tabellen
\usepackage[format=default,font=footnotesize,labelfont=bf]{caption} % Kümmert sich um Darstellung von Bildüberschriften
\usepackage{listings}
\usepackage{color} % Bindet verschiedene Schriftfarben ein.
\usepackage[pdftex,dvipsnames,table,xcdraw]{xcolor}  % Noch mehr Farben.
\usepackage{float} % Man kann für Fließbilder und Tabellen verschiedene Stylings auswählen.

\definecolor{CiteColor}{rgb}{0,0,0.6}
\definecolor{LinkColor}{rgb}{0.6,0,0}
\usepackage[
    bookmarks,
    bookmarksopen=true,
    colorlinks=true,
    linkcolor=LinkColor, 
    citecolor=CiteColor, 
    %linkcolor=black,
    %citecolor=black,
    menucolor=red, 
    anchorcolor=black,
    urlcolor=black,
    filecolor=black,
    backref,
    plainpages=false, 
    pdfpagelabels,
    hypertexnames=false, 
    linktocpage=true
    ]{hyperref}
   % ermöglicht komfortable Referenzierung im Dokument (autoref)
\usepackage[printonlyused]{acronym} % Funktionen und Befehle für Abkürzungsverzeichnis
\usepackage[sort&compress,square,comma,authoryear]{natbib}
\usepackage{bibtopic} % erstellt Literaturverzeichnis
% bibtopic needs a special compilation. use pdflatex -synctex=1 -interaction=nonstopmode %.tex|bibtex %1.aux|bibtex %2.aux|bibtex %3.aux|pdflatex -synctex=1 -interaction=nonstopmode %.tex|pdflatex -synctex=1 -interaction=nonstopmode %.tex in fast translation when in TexMaker
%\usepackage{bpchem}
\usepackage{textcomp}
\usepackage{pdflscape}
\usepackage[colorinlistoftodos,prependcaption]{todonotes}
\usepackage{xargs}   % Use more than one optional parameter in a new commands
\usepackage{amsmath}
\usepackage{multirow}
\usepackage{fancyvrb} % style verbatims
\usepackage{amssymb}
\usepackage{nicefrac} % schöne Brüche im Textfluss
\usepackage{rotating}



