%%% ABSTRACT GERMAN %%%	
\section*{Zusammenfassung}
\begin{spacing}{1.2}
\acused{XOR}
In dieser Masterarbeit wird ein neuartiger \ac{MitM} Angriff auf eine Sprachverbindung im \ac{GSM} Netz entwickelt. Der Angriff nutzt den fehlenden Integritätsschutz auf der Funkschnittstelle aus, um in den verschlüsselten Anrufaufbau eines Opfers einzugreifen und den Anruf an ein Mobiltelefon unter Kontrolle des Angreifers umzuleiten. Im theoretischen Teil der Arbeit wird der Angriff entwickelt, seine Machbarkeit mathematisch nachgewiesen und die zugrunde liegende Schwachstelle beschrieben. Im praktischen Teil wird der Angriff für einen \ac{MitM}, in einer Testumgebung mit einer virtualisierten Funkschnittstelle, implementiert und durchgeführt. Die in \ac{GSM} verwendeten Verschlüsselungsverfahren sind Stromchiffren, die die Vertraulichkeit des Datenstroms durch die \ac{XOR}-Kombination mit einem Schlüsselstrom schützen. Es wird gezeigt dass bekannte Teile des Chiffrestroms beliebig manipuliert werden können. Da \ac{GSM} kein Verfahren für den Integritätsschutz spezifiziert, kann der Angreifer die Telefonnummer im ausgehenden Anruf also unbemerkt ersetzen und den Anruf an ein von ihm bestimmtes Endgerät umleiten. Durch die Verknüpfung des eingehenden Anrufs mit einem neuen Anruf bei der ersetzten Nummer, erhält der Angreifer Zugriff auf das geführte Gespräch -- ein \ac{MitM}-Angriff auf die Sprachverbindung. Das Netzwerk und die Endgeräte kümmern sich für den Angreifer um Verschlüsselung und Kodierung der Sprachdaten, da es sich um reguläre \ac{GSM}-Anrufe handelt. Der Angriff ist selbst bei der Verwendung von sicheren Verschlüsselungsverfahren, wie A5/4 anwendbar, da die Verschlüsselung nicht gebrochen werden muss. Auch das aktuelle 3G \ac{AKA} bietet keinen Schutz, da es in \ac{GSM} zwar die gegenseitige Authentifizierung ermöglicht, aber keinen Integritätsschutz unterstützt. Der Angriff erfordert die teilweise Kenntnis der vom Opfer angerufenen Telefonnummer ("`Known Plaintext"'). Diese kann als gegeben angenommen werden, da die Rufnummern in der Regel bekannte Teile aufweisen. \citet{meyer2004man} zeigten, dass es möglich ist, ein \ac{MitM}-Gerät auf der Funkschnittstelle zu installieren, was für die Arbeit vorausgesetzt und nicht näher untersucht wird. Für die praktische Durchführung des Angriffs wird die physikalische Ebene der Funkschnittstelle, auf Basis von Multicast-Sockets, virtualisiert und für diese ein \ac{MitM} implementiert. Die Durchführung des Angriffs erfolgt im Rahmen des Osmocom Projekts. Im letzten Teil der Arbeit wird Bezug zu verwandten Angriffen auf den \ac{GSM}-Standard aufgebaut und die Vorteile gegenüber diese herausgearbeitet.
\end{spacing}
\vfill
\acresetall