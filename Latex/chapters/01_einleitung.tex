%%%%%%%%%%%%%%%%%%%%%%%%%%%%%%%%%%%%%%%%%%%%%%%%%%%%%%%%%%%
%%% Einleitung
%%%%%%%%%%%%%%%%%%%%%%%%%%%%%%%%%%%%%%%%%%%%%%%%%%%%%%%%%%%
\chapter{Einleitung} \label{hdl:einleitung}

Seit der Einführung von \ac{GSM} 1992 hat sich der Mobilfunk stetig weiterentwickelt. So wurden neue, schnellere und sicherere Standards für Mobilfunknetzwerke spezifiziert und ausgerollt. Weil das \ac{GSM}-Netz aber wegen seiner bestehenden und weit verbreiteten Infrastruktur flächendeckenden Empfang liefert und kostengünstig ist, wird es in den meisten Regionen trotz Ausbau der 3G und 4G Netze weiterhin unterstützt und verwendet \citep{opensignal}. In Japan, Korea und Singapur ist die Abschaltung des 2G Netzes bereits erfolgt und in Australien bis September 2017 geplant. Das Ziel dabei ist vor allem die Wiederverwendung der begehrten, aber von \ac{GSM} belegten Frequenzen im 900 MHz Band. In Europa wird laut \citet{heise:newsticker-3582914} aktuell die Unterstützung und Instandhaltung der Infrastruktur von \ac{GSM} bis etwa 2020 vorausgesagt - viele Netzanbieter halten sich diesbezüglich bedeckt. Da vor allem im \ac{IoT} Bereich und für die \ac{M2M} Kommunikation viele \ac{GSM}-Module verwendet werden, die die neueren Standards nicht unterstützen, hält man sich in einigen Ländern wie Deutschland mit konkreten Angaben ganz zurück. Der Lebenszyklus dieser meist im Embedded-Bereich verwendeten Geräte ist in der Regel hoch und ohne \ac{GSM} müssten sie ersetzt werden, was hohe Kosten verursachen würde. Auch 2017 wird noch in den Ausbau von \ac{GSM}-Netzwerken investiert. In Mexiko wird zum Beispiel für schwer erreichbare Regionen von Non-Profit-Organisationen der \ac{GSM} Netzausbau auf Basis von Open-Source-Software des Osmocom Projektes vorangetrieben \citep{osmocom:news-rhizomatica}. Es wird derzeit angenommen, dass das 3G Netz als Übergangstechnologie vom sprachbasierten \ac{GSM} zum datenbasierten \ac{LTE} früher abgeschaltet wird als \ac{GSM}. Die \ac{GSM}-Infrastruktur wird, wenn man von der derzeitigen Entwicklung ausgeht, also noch lange erhalten bleiben. Die folgenden Zitate stützen diese Behauptung.

\textit{"`In terms of global reach, cellular networks already cover 90 percent of the world’s population.
WCDMA and LTE are catching up, but GSM will offer superior coverage in many markets for
years to come."'} \citep{Ericsson:2016:Uen:284-23-3278}

\textit{"`Speziell für M2M Anwendungen könnte GSM über 2020 hinaus weiterhin relevant bleiben."'} \citep[Tom Tesch]{heise:newsticker-3582914}

\textit{"`We are still maintaining our old networks, and modernising the network, also still delivering 2G and 3G services to the customer. And yes, 2G will continue longer than we expected"'} \citep[Matthias Sauder, Vodafone]{mobileworldlive:long-life-2g}

Der wohl noch länger andauernden Laufzeit von \ac{GSM} stehen dessen veraltete Sicherheitsmechanismen und die Einstellung von Mobilfunkanbietern und Herstellen, diese zu überarbeiten gegenüber. Harald Welte fasst diese in seinem Blog zum Thema Sicherheit zusammen.

\textit{"`GSM equipment manufacturers and mobile operators have shown no interest in fixing gaping holes in their security system."'} \citep[Harald Welte]{laforge:blog-20101112}

So wird \ac{GSM} ohne größere Veränderungen auf dem gleichen Stand der Technik wie vor 30 Jahren betrieben. Trotz einiger Versuche die Sicherheit zu verbessern, wie die Einführung der gegenseitigen Authentifizierung oder die Spezifikation von neuen Verschlüsselungsalgorithmen wie A5/4, bleibt \ac{GSM} verwundbar und anfällig für eine Vielzahl von Angriffen. Gründe dafür sind Schwachstellen in den Anpassungen der Sicherheitsmechanismen und die verzögerte Einführung dieser von den Netzanbietern. Das auch für \ac{GSM} spezifizierte \ac{UMTS}-\ac{AKA} gewährleistet in \ac{GSM} zum Beispiel gegenseitige Authentifizierung, aber keinen Integritätsschutz, eine Schwachstelle in der Spezifikation. Der in dieser Arbeit vorgestellte \ac{MitM}-Angriff nutzt diese Schwachstelle, sowie in \ac{GSM} unverschlüsselt preisgegebene Informationen, um in den Anrufaufbau einzugreifen und ein Telefonat zu einem Angreifer umzuleiten. Der Eingriff liegt in der Manipulation der angerufenen Telefonnummer in der verschlüsselten, für den Anrufaufbau zuständigen Signalisierungsnachricht. Der Angriff verdeutlicht die Notwendigkeit, Möglichkeiten für den Schutz der Integrität für den \ac{GSM}-Standard zu spezifizieren und in den \ac{GSM}-Netzwerken einzuführen. Verschlüsselung ohne Integritätsschutz stellt eine gravierende Schwachstelle dar. Vor allem im Hinblick auf die voraussichtlich noch lange Laufzeit von \ac{GSM}, muss diese dringend behoben werden.