%%%%%%%%%%%%%%%%%%%%%%%%%%%%%%%%%%%%%%%%%%%%%%%%%%%%%%%%%%%
%%% Ausblick
%%%%%%%%%%%%%%%%%%%%%%%%%%%%%%%%%%%%%%%%%%%%%%%%%%%%%%%%%%%
\chapter{Zusammenfassung und Ausblick} \label{hdl:ausblick}
Der in dieser Arbeit vorgestellte, praktisch durchführbare \ac{MitM}-Angriff auf \ac{GSM} ist der erste seiner Art. Aktuelle existierende \ac{MitM}-Angriffe, die in den Datenverkehr auf der Funkschnittstelle eingreifen, basieren auf der Verwendung von schwachen Verschlüsselungsverfahren über die der kryptografische Schlüssel gewonnen und die Verschlüsselung gebrochen werden kann. Der entwickelte \ac{MitM}-Angriff hingegen basiert rein auf der Schwachstelle des fehlenden Integritätsschutzes in \ac{GSM} und kann unabhängig von verwendeten Verschlüsselungsverfahren durchgeführt werden. Wie schon bei dem von \citet{meyer2004man} veröffentlichten \ac{MitM}-Angriff schützt auch die Verwendung des gegenseitigen \ac{UMTS}-Authentifizierungsverfahrens nicht, da es für \ac{GSM} keinen Integritätsschutz bietet.

Der Angriff kombiniert zwei \ac{MitM}-Angriffe. Beim ersten wird vorausgesetzt, dass die Kommunikation zwischen dem Mobiltelefon des Opfers und dem Netzwerk über ein vom Angreifer kontrolliertes Gerät auf der Funkschnittstelle läuft. Dieses Gerät kann entweder eine falsche Basisstation sein, oder rein auf physikalischer Ebene arbeiten. Ein \ac{MitM}-Gerät auf physikalischer Ebene wäre zum Beispiel ein modifizierter \ac{GSM}-Repeater. Der Angreifer kann aus der unverschlüsselten, vor dem Anrufaufbau notwendigen Kommunikation zwischen einem Netzteilnehmer und dem Netzwerk Rückschlüsse auf die Identität des Anrufers und das exakte Frame, in dem die Setup-Nachricht gesendet wird, schließen. Die Setup-Nachricht enthält Informationen über den ausgehenden Anruf, unter anderem auch die angerufene Telefonnummer. Die Setup-Nachricht ist in der Regel gleich aufgebaut, womit sich die Telefonnummer an einer dem Angreifer bekannten Position befindet. Mit den vorhandenen Informationen kann der Angreifer die Setup-Nachricht eines vom Opfer ausgehenden Anrufs eindeutig identifizieren. Die Nachricht enthält kodierte und verschlüsselte Daten, ist also ohne Kenntnis des kryptografischen Schlüssels nicht ohne weiteres lesbar. Da die verwendeten Kodierungsverfahren bekannt sind, eine Stromverschlüsselung benutzt wird und kein Integritätsschutz vorhanden ist, kann der Angreifer aber bekannte Originaltextteile beliebig ändern. Der Angriff macht sich das zu Nutze, um die angerufene Nummer zu ändern und den Anruf an ein Mobiltelefon unter Kontrolle des Angreifers weiterzuleiten. Es wird vorausgesetzt, dass die angerufene Telefonnummer oder Teile dieser vom Angreifer in Erfahrung gebracht wurde, zum Beispiel durch Social Engineering. Der Angreifer verknüpft nun den eingehenden Anruf mit einem neuen Anruf bei der ersetzten Nummer und leitet kommunizierte Gesprächsdaten zwischen diesen weiter. Das Ergebnis ist ein \ac{MitM}-Angriff mit Zugriff auf den unverschlüsselten Gesprächsverkehr zwischen zwei Opfern. Die Daten können unabhängig vom verwendeten Verschlüsselungsverfahren abgehört, aufgenommen und manipuliert werden. Zudem besteht nach dem Anrufaufbau keine Abhängigkeit mehr zum \ac{MitM}-Gerät auf der Funkschnittstelle. Ein Handover zu einer anderen \ac{BTS} unterbricht den \ac{MitM}-Angriff nicht, Opfer und Angreifer können sich frei bewegen.

Der Angriff wurde in dieser Arbeit sowohl theoretisch ausgearbeitet, als auch praktisch in einer virtuellen Testumgebung durchgeführt. Im theoretischen Teil wurde zuerst der für den Anrufaufbau notwendige Nachrichtenverkehr sowie der Inhalt der Setup-Nachricht analysiert. Im Anschluss wurde die Möglichkeit, bekannte Daten in einer \ac{GSM} kodierten und stromverschlüsselten Nachricht zu manipulieren, mathematisch nachgewiesen. Für die praktische Verifikation des Angriffs wurde freie Software aus dem Osmocom Projekt verwendet. Um hardwareunabhängig testen zu können, wurden die Projekte osmoBTS (die Basisstation) und osmocomBB (das Mobilfunkgerät) um eine virtuelle Luftschnittstelle erweitert. Die physikalische Ebene der Funkschnittstelle wurde dafür durch Multicast-Sockets ersetzt, die \ac{GSMTAP} gekapselte Nachrichten austauschen. Für den \ac{MitM}-Angriff wurde eine \ac{MitM}-Implementierung für das virtuelle \ac{Um} geschrieben, die einen \ac{IMSI}-Catcher für die Identifikation des Opfers sowie Funktionalität für die Identifikation und Manipulation der kodierten, verschlüsselten Setup-Nachricht umsetzt. Der Angriff konnte damit erfolgreich durchgeführt und seine praktische Machbarkeit so verifiziert werden.

Was diese Arbeit offenlässt, ist die Durchführung des kompletten vorgestellten \ac{MitM}"=Angriffs. Die Verknüpfung der Audiodaten der beiden Gesprächspartner für den \ac{MitM}-Angriff beim Mobiltelefon des Angreifers sollte aber keine große Hürde darstellen. Die Machbarkeit des \ac{MitM}-Angriffs auf der \ac{Um}-Schnittstelle wurde in einer Testumgebung nachgewiesen. Mit den Erkenntnissen aus dieser Arbeit und basierend auf den Implementierungen wäre der nächste Schritt die praktische Durchführung des Angriffs auf der realen Funkschnittstelle. Interessant wäre dabei insbesondere die Verwendung eines \ac{GSM}-Repeaters als \ac{MitM}, da die Eignung und Möglichkeiten dieser Geräte dahingehend noch nicht untersucht wurde. Auf Basis der durch Timing Advance gewonnenen Zeitspanne für die Echtzeitverarbeitung und Änderung von Daten, sollte er aber möglich sein. Zuletzt wird in dieser Arbeit vorausgesetzt, dass die Länge des Felds für die Bearer-Capability in der Setup-Nachricht bekannt ist. Eine unbekannte Länge würde ein unbekanntes Offset der Telefonnummer bedeuten und deshalb ein Problem für den Angriff darstellen. Das Feld wird für die Übermittlung von unterstützten Sprachkodierungen und weiteren, anrufbezogenen Einstellungen verwendet. Die Länge des Feldes wird im Standard als dynamisch spezifiziert, in Mitschnitten der Testumgebung wurde sie jedoch als konstant beobachtet. Wegen der Funktion des Feldes wird angenommen, dass sein Inhalt vom Telefonmodell eines Herstellers abhängt und so durch die Abfrage der \ac{IMEI} des Gerätes oder durch Social-Engineering in Erfahrung gebracht werden kann. Diese Annahme muss durch ausreichende Tests mit verschiedenen Mobilfunkgeräten noch nachgewiesen werden.

Das Fazit der Arbeit ist allgemein bekannt \citep{yu2004perils}\citep{paterson2006cryptography}\citep{degabriele2007attacking}\citep{bittau2006final}: Verschlüsselung ohne Integritätsschutz gewährleistet keine Vertraulichkeit von Daten. Für \ac{GSM} bedeutet das im speziellen, dass auf der \ac{Um}-Schnittstelle übertragene Daten nicht vor Angriffen geschützt sind, auch wenn eigentlich sichere Verschlüsselungsverfahren wie A5/4 verwendet werden. \ac{GSM} bleibt solange verwundbar, bis Methoden für den Schutz der Datenintegrität spezifiziert und in den Netzwerken verwendet werden. Die Umsetzung eines praktischen, von der Verschlüsselung unabhängigen \ac{MitM}-Angriffs in dieser Arbeit verdeutlicht diese gravierende Sicherheitsschwachstelle des \ac{GSM}-Netzwerks.