\chapter{Verwandte Arbeiten und Angriffe} \label{hdl:einleitung_stand-der-technik}

Neben dem hier vorgestellten Angriff gibt es eine Reihe von weiteren Angriffen auf das \ac{GSM}-Netzwerk, die Schwachstellen der \ac{GSM}-Protokolle ausnutzen. Passive Angriffe lauschen nur und greifen nicht in den Datenverkehr ein, bei aktiven Angriffen kann der Angreifer den Datenverkehr manipulieren. In diesem Kapitel wird auf passive und aktive Angriffe eingegangen, die in Bezug zur Arbeit stehen.

\section{Passive Angriffe} \label{hdl:passive-attacks}
Die Angriffe in diesem Abschnitt beziehen sich auf Schwachstellen der \ac{GSM}-Protokolle, die es einem passiven Angreifer ermöglichen, an Daten zu gelangen, die nicht für ihn bestimmt sind.

Die erste Kryptoanalyse des A5/1 Algorithmus wurde drei Jahre, nachdem sein Design bekannt wurde, veröffentlicht. \citet{golic1997cryptanalysis} legte darin mehrere Schwachstellen offen und stellte die Idee eines Time-Memory-Trade-Off Angriffs vor, der auf dem Geburtstagsparadoxon beruht. \citet{Biryukov2001} analysierten den Angriff mit dem Ergebnis, dass er nicht praktikabel war. Er hätte für die Durchführung 15 TB vorberechnete Daten und mehrere Stunden bekannter Gesprächsdaten benötigt. \citet{Biryukov2001} entwickelten den Angriff in ihrer Arbeit weiter, indem sie mehrere weitere Schwachstellen des A5/1 Algorithmus ausnutzten. Ihr vorgestellter Angriff braucht bis zu 290 GB vorberechneter Daten und hat eine Ausführungszeit von etwa einer Sekunde. Der Nachteil liegt in der mit bis zu zwei Minuten relativ langen Sequenz von Gesprächsdaten, die bekannt sein müssen. Eine weitere Verbesserung des Angriffs auf A5/1 entwickelten \citet{barkan2003instant}. Der Angriff nutzt durch Fehlerkorrekturmechanismen generierte, bekannte Redundanz auf dem \ac{TCH} aus und kommt deshalb ohne die Voraussetzung von bekannten Gesprächsdaten aus. Die Dauer des Angriffs wurde außerdem auf unter eine Sekunde reduziert und kommt damit Echtzeitentschlüsselung nahe. Der Nachteil ist die mit mehreren Terabyte große Menge an Daten, die für den Angriff vorberechnet werden müssen. Für die Zeit der Veröffentlichung wäre mit der Berechnung der Daten für die Entschlüsselung eines 5 Minuten Gesprächs ein handelsüblicher Computer mehrere Jahre beschäftigt gewesen, was die praktische Anwendung unrealistisch machte. Auf einer Blackhat Präsentation erwähnte  \citet{hulton2008intercepting} die Arbeit an der Berechnung von Rainbow-Tables für einen Time-Memory-Trade-Off Echtzeitangriff auf A5/1. \citet{nohl2009gsm} stellte auf der 26C3 den ersten praktisch durchführbaren Angriff auf A5/1 vor. Er schätzte, dass die Berechnungsdauer der Rainbow-Tables von 100.000 Jahren auf einem Prozessor, durch parallele Programmierung mit CUDA\footnote{\url{http://www.nvidia.de/object/cuda-parallel-computing-de.html}} und Ausführung auf 80 \acp{GPU}, auf 3 Monate reduziert werden kann. Auf der 27C3 führte er den Angriff mit Munaut praktisch durch \citep{nohl2010wideband}. Die Größe der für den Angriff verwendeten Rainbow-Tables konnte auf 2 TB und deren Berechnungsdauer auf einen Monat mit vier \acp{GPU} reduziert werden. Im Anschluss an den Vortrag wurden die Tables veröffentlicht\footnote{\url{https://opensource.srlabs.de/projects/a51-decrypt}}. Die Erfolgsrate des Angriffs wird von Nohl auf 99\% geschätzt, sofern die Registrierung des Nutzers am Netzwerk mit aufgenommen werden kann, da die bekannten Signalisierungsnachrichten dann maximal sind. Andernfalls sinkt die Erfolgschance auf 50\%. Mit dem von Nohl vorgestellten Angriff und den veröffentlichten Rainbow-Tables ist es möglich, einen belauschten und A5/1 geschützten Anruf zu entschlüsseln. Für das Abhören des Telefonats nutzten Nohl und Munaut in ihrer Präsentation zwei Motorola C123 Mobiltelefone (2017 ca. 30 Euro) mit angepasster Firmware aus osmocomBB. Um die Nachrichten auf der Funkschnittstelle zu belauschen, kann auch ein \ac{USRP} (Ettus Research ca. 1500 Euro) oder ein \ac{SDR} wie HackRF (ca. 300 Euro) und freie Software\footnote{\url{http://cgit.osmocom.org/gr-osmosdr/}} genutzt werden.

A5/2 wurde mit der Einführung von \ac{GSM} für Exportregionen entwickelt und bietet signifikant geringeren Schutz als der zu dieser Zeit als sicher geltende A5/1. Seine kryptografische Schwäche beruht auf einer geringeren verwendeten Schlüssellänge als in A5/1. \citet{goldberg1999real} veröffentlichten einen Angriff, mit dem das Verschlüsselungsverfahren in Echtzeit gebrochen werden kann. A5/2 wurde einige Jahre, nachdem ein praktischer Angriff von \citet{barkan2003instant} veröffentlicht wurde, von \ac{3GPP} offiziell aus der Liste der unterstützten Algorithmen genommen \citep{osmocom:withdrawal-a52}.

\citet{dunkelman2010practical} publizierten einen praktischen Angriff auf den KASUMI Algorithmus, der in A5/3 für die Generierung des Schlüsselstroms verwendet wird. Wegen der Art der Verwendung von KASUMI ist der veröffentlichte Sandwich-Angriff aber nicht auf A5/3 anwendbar. KASUMI arbeitet mit einem 128 Bit Schlüssel, da der kryptografische Schlüssel in \ac{GSM} nur 64 Bit hat, geht dieser also konkateniert ($\ac{Kc} \parallel \ac{Kc}$) in KASUMI ein. Durch die resultierende Entropie des Schlüssels von nur 64 Bit ist A5/3 generell anfällig für Bruteforce Angriffe. \citet{nohl2014mobile} bezeichnete A5/3 auf der 31C3 als "`NSA verwundbar"'. Wegen seiner geringen Schlüssellänge kann der Algorithmus Angreifern mit entsprechenden finanziellen Mitteln nicht widerstehen. Laut \citet{theintercept:nsa-auroragold} arbeitet die \ac{NSA} daran, den A5/3 Algorithmus zu brechen. Auch wenn noch kein praktischer Angriff auf A5/3 veröffentlicht wurde, sollte man sich auf längere Sicht nicht auf dessen Sicherheit verlassen.
\begin{sloppypar}
Um das Problem der geringen Schlüssellänge zu beheben, wurde A5/4 spezifiziert \citepauthor{3gpp:55.226}. Dieser nimmt volle 128 Bit für den in KASUMI verwendeten Schlüssel entgegen. Sofern eine 3G \ac{USIM}-Karte verwendet wird, die einen Schlüssel mit dieser Entropie bereitstellt, ist A5/4 nicht anfällig für Bruteforce Angriffe. 2G \ac{SIM}-Karte Karten werden von Netzanbietern nicht mehr vertrieben und praktisch nicht mehr verwendet, allerdings gibt es in Deutschland auch noch keinen Mobilfunkanbieter, dessen \ac{GSM}-Netz A5/4 unterstützt \citep{gsmmap:secrep-ger}.
\end{sloppypar}
Neben den Schwachstellen der Algorithmen gibt es in \ac{GSM} das Problem, dass Mobiltelefone auch unverschlüsselte Verbindungen akzeptieren, wenn das Netzwerk keine Verschlüsselung bereitstellt. Für passive Angriffe ist die Schwachstelle nicht relevant, da \ac{BTS} von Netzanbietern Verschlüsselung verlangen, für aktive Angriffe ist sie aber von Bedeutung. 

Neben der Verschlüsselung stellt auch Frequency Hopping ein Problem beim Aufzeichnen, Abhören und Manipulieren von Datenverbindungen dar. Der Angreifer muss die verwendete Sprungsequenz der Frequenz kennen und ihr folgen, um den Datenverkehr aufzeichnen zu können. \citet{nohl2010wideband} bemängelten, dass Frequency Hopping in den meisten Fällen nicht genutzt wird. In seiner Präsentation setzte Nohl den Punkt auf seine Wunschliste an die Netzanbieter. Aus Untersuchungen geht hervor, dass in Deutschland zum Stand 2016 Frequency Hopping zwar in der Regel genutzt wird, die verwendeten Sprungsequenzen aber oft vorhersehbar sind. Auch weitere angemerkte Sicherheitsverbesserungen, wie zufälliges Padding, häufiger Wechsel der \ac{TMSI} und das Aushandeln eines neuen kryptografischen Schlüssels für jeden genutzten Dienst, sind in Deutschland bis 2016 kaum umgesetzt worden \citep{gsmmap:secrep-ger}.

\section{Aktive Angriffe}
Die Angriffe in diesem Abschnitt beziehen sich auf Schwachstellen im \ac{GSM}-Protokoll, die es einem aktiven Angreifer ermöglichen, den Nachrichtenverkehr so zu manipulieren, dass er an Informationen kommt, die nicht für ihn bestimmt sind.

Die Grundlage für die meisten vorgestellten aktiven Angriffe ist ein Designfehler im \ac{GSM}-Authentifizierungsverfahren, der es einem Angreifer ermöglicht, sich Mobiltelefonen gegenüber als \ac{BTS} auszugeben. Durch das einseitige \ac{GSM}-\ac{AKA} kann eine \ac{MS} die Authentizität der \ac{BTS} nicht überprüfen. Diese falschen \ac{BTS} wurden erstmals in \citet{gobel1996strafprozess}, als \ac{IMSI}-Catcher erwähnt. Das ursprünglich von Rohde{\&}Schwarz\footnote{\url{https://www.rohde-schwarz.com}} entwickelte Gerät "`GA 090"' wurde von Behörden eingesetzt, um die \acp{IMSI} der Netzteilnehmer in Reichweite zu sammeln, wodurch deren Identität und Standort ermittelt werden konnte. Das Sammeln der Identitäten ist durch eine weitere Schwachstelle im \ac{GSM}-Protokoll möglich. In der Regel wird die \ac{IMSI} der Teilnehmer auf dem \ac{Um} verschleiert und durch eine \ac{TMSI} ersetzt. Die \ac{IMSI} kann aber jederzeit vom Netzwerk angefordert werden, falls dieses die Zuordnung von \ac{IMSI} zur \ac{TMSI} verliert. Da das Netzwerk die Identität und damit den kryptografischen Schlüssel des Teilnehmers nicht kennt, ist der "`Identity Request"', mit dem die \ac{IMSI} abgefragt wird, unverschlüsselt. In \ac{GSM} ist außerdem spezifiziert, dass eine \ac{MS} während einer \ac{RR}-Verbindung einen Identity Request des Netzwerks jederzeit beantworten muss \citepauthor[Kap. 4.3.3.2]{3gpp:24.008}. Ein solcher Identity Request kann vom Angreifer direkt beim von der \ac{MS} initiierten \ac{RR}-Verbindungsaufbau geschickt werden, um an die \ac{IMSI} zu kommen. Durch geringe Modifikationen an der Betriebssoftware des Geräts, konnte der \ac{IMSI}-Catcher auch Nachrichten zwischen den \ac{MS} und der echten \ac{BTS} des Netzanbieters abhören und weiterleiten \citep{fox2002imsi}.

Für ihren vorgeschlagenen \ac{MitM}-Angriff zum Abhören eines verschlüsselten Telefonats setzten \citet{barkan2003instant} eine falsche \ac{BTS} mit Zugriff auf den Nachrichtenverkehr zwischen \ac{MS} und Netzwerk voraus. Der Angreifer ist über diese auf der einen Seite mit dem Netzwerk und auf der anderen mit dem Opfer verbunden. Um die falsche \ac{BTS} als das Opfer beim Netzwerk authentifizieren zu können, wird die Authentifizierungsanfrage des Netzwerks an die \ac{MS} des Opfers weitergeleitet, die \ac{SRES} berechnet und dem Angreifer zurückschickt. Mit \ac{SRES} kann die falsche \ac{BTS} sich nun ihrerseits beim Netzwerk authentifizieren und sich ihm gegenüber als Opfer ausgeben. Der Angreifer kann die Verschlüsselung zwischen falscher \ac{BTS} und \ac{MS} selbst festlegen, unabhängig von der Verschlüsselung, die ihm vom Netzwerk vorgegeben wird. Wählt er einen schwachen Algorithmus wie A5/2, so kann er dessen Schwachstellen ausnutzen, um den von der \ac{MS} des Opfers generierten kryptografischen Schlüssel \ac{Kc} zu gewinnen und damit die Daten des Opfers entschlüsseln. Da der selbe kryptografische Schlüssel \ac{Kc} für alle \ac{A5} Algorithmen verwendet wird, können die Daten des Opfers für das Netzwerk mit jeder beliebigen, geforderten Verschlüsselung verschlüsselt werden.

\citet{nohl2009gsm} zeigten auf der 26C3, dass es mit günstiger Hardware und freier Software möglich ist, einen \ac{IMSI}-Catcher umzusetzen. Die praktische Umsetzung des Angriffs führte \citet{paget2010practical} ein Jahr später auf der DefCon vor. Er verwendete einen \ac{USRP}, der über seinen Laptop mit OpenBTS verbunden war, als \ac{BTS} und Asterisk, sowie Wireshark zum Aufzeichnen und Dekodieren des Datenverkehrs. Er brachte Mobiltelefone dazu, sich mit A5/0, also ohne Verschlüsselung, mit der falschen \ac{BTS} zu verbinden. Des Weiteren führte er vor, dass durch Stören der \ac{UMTS}-Frequenzen, \acp{MS} trotz Anwesenheit eines 3G Netzes dazu gezwungen werden können, sich auf den \ac{GSM}-Frequenzen mit der falschen \ac{BTS} zu verbinden. Auch den \ac{MitM}-Angriff von \citet{barkan2003instant} konnte er mit seiner falschen \ac{BTS} praktisch vorführen und die schwachen Algorithmen A5/1 und A5/2 für die Verschlüsselung aushandeln.

Mit dem \ac{UMTS}-Standard wurde ein gegenseitiges Authentifizierungsverfahren eingeführt, um \ac{MitM}-Angriffe zu verhindern. Der Datenverkehr auf der \ac{Um}-Schnittstelle wird in \ac{UMTS} mit einem Integritätsschlüssel \ac{IK} geschützt und das Netzwerk authentifiziert sich mit \ac{AUTN} beim Mobiltelefon. \citet{meyer2004man} stellten jedoch Szenarien vor, in denen ein \ac{MitM}-Angriff trotz der \ac{UMTS}-Sicherheitsmechanismen möglich ist. Die Voraussetzung dafür ist, dass das Telefon des Opfers zusätzlich zu \ac{UMTS} das \ac{GSM}-Netzwerk unterstützt und eine \ac{GSM}-\ac{BTS} in Reichweite ist. Der parallele Betrieb von \ac{GSM} und \ac{UMTS}-Infrastruktur ist auch 2017 noch meistens der Fall. Der Angriff macht sich zu Nutze, dass in \ac{GSM}-Netzwerken auch bei Verwendung des \ac{UMTS}-\ac{AKA} kein Integritätsschutz unterstützt wird. Der Angreifer kann so an die Authentifizierungsparameter \ac{RAND} und \ac{AUTN} des Netzwerks gelangen und sich der \ac{MS} gegenüber als dieses ausgeben. Wie in den vorherigen Fällen kann dann eine schwache Verschlüsselung gefordert und ein \ac{MitM}-Angriff ausgeführt werden.

Neben der Identifikation von Opfern und Angriffen die darauf abzielen verschlüsselte Kommunikation mit dem Netzwerk abzuhören, gibt es eine Reihe von weiteren Anwendungsfällen für eine \ac{MitM}-\ac{BTS} auf der Funkschnittstelle. So kann durch binäre \ac{SMS} mit \ac{OTA} Befehlen Schadcode auf der \ac{JVM} von \ac{USIM}-Karten ausgeführt werden. Dieser kann geheime Daten von Bankanwendungen oder den geheimen Schlüssel des Opfers auslesen und dem Angreifer über \ac{SMS} zuschicken \citep{nohl2013rooting}. Mit speziellen Nachrichten kann ein Angreifer außerdem auf die Kontrollschnittstelle für die Fernwartung von Mobiltelefonen zugreifen. Damit ist es zum Beispiel möglich, einen neuen \ac{APN} oder \ac{HTTP}-Proxy einzustellen und so einen permanenten \ac{MitM} einzurichten \citep{solnik2014cellular}.

Die Notwendigkeit von Integritätsschutz für verschlüsselte Daten ist bekannt. Daraus resultierende Sicherheitsprobleme wurden zum Beispiel von \citet{yu2004perils} für den Authentifizierungsdienst Kerberos behandelt. \citet{paterson2006cryptography} wiesen auf die selben Probleme in der \ac{IPSec}-Implementierung im Linux Kernel hin, woraufhin \citet{degabriele2007attacking} eine Reihe implementierungsunabhängiger Angriffe auf die \ac{IPSec}-Spezifikation entwickelten. Von besonderem Interesse für diese Masterarbeit ist der von \citet{bittau2006final} publizierte Fragmentierungsangriff auf die Stromverschlüsselung von \ac{ESP}-Paketen in \ac{WEP}. Der 802.11 Standard spezifiziert, dass ein großes Paket auch fragmentiert mit mehreren kleinen Paketen übertragen werden kann. Bittau und Handley konnten durch den bekannten Klartext des Protokollheaders 8 Byte des Schlüsselstroms extrahieren. Da der 802.11 Standard die Wiederverwendung des Schlüsselstroms für bis zu 16 aufeinanderfolgende Pakete erlaubt, konnten in Kombination mit der Fragmentierung bis zu 64 Byte an Daten eingeschleust werden. Die 8 Byte Fragmente wurden alle mit dem bekannten Teil des Schlüsselstroms verschlüsselt. Das besondere am Angriff ist, dass er nicht darauf abzielt, den kryptografischen Schlüssel herauszufinden, der für die Generierung des Schlüsselstroms verwendet wird. Stattdessen nutzt er einen bekannten Teil des Schlüsselstroms und Schwachstellen des Protokolls aus, um den stromverschlüsselten Datenverkehr zu manipulieren. Der in dieser Arbeit vorgestellte Angriff fällt in die gleiche Kategorie.

Statt einen \ac{MitM}-Angriff über eine falsche \ac{BTS} auf der Funkschnittstelle durchzuführen, kann ein Angreifer das Netzwerk auch direkt angreifen. So zeigten \citet{golde2012weaponizing}, dass ein Angreifer sich Zugang zu einer Femtozelle verschaffen und diese so manipulieren kann, dass sie für oben erwähnte \ac{MitM}-Angriffe, wie Verkörperung eines Nutzers und Abhören und Modifizieren des Datenverkehrs, genutzt werden kann. Eine Femtozelle ist eine typischerweise in Privat- oder Firmenumgebungen installierte \ac{BTS} mit geringer Reichweite, die über einen \ac{DSL} Anschluss direkt mit dem Netzwerk des Netzanbieters verbunden ist. Ein weiterer Ansatz ist der direkte Angriff auf das Kernnetzwerk des Netzanbieters. Die Sicherheit dieses \ac{SS7}-Netzwerks basiert auf dem gegenseitigen Vertrauen aller im Netzwerk agierenden Komponenten, der Datenverkehr wird nicht weiter geschützt. Ein 2014 erschienener Artikel der Washington Post\footnote{"`For sale: Systems that can secretly track where cellphone users go around the globe"' (Timberg,
2014)} zeigte unter Referenz der Broschüre eines Überwachungsdienstleisters, dass das Szenario eines Angreifers mit Zugriff auf das \ac{SS7}-Netzwerk Realität ist. Daraufhin beschäftigten sich unter anderem \citet{nohl2014mobile} und \citet{Mourad:fall-of-ss7} mit den Konsequenzen für die Sicherheit des Mobilfunks und stellten eine Vielzahl von \ac{MitM}-Angriffen vor, die den Zugang zum \ac{SS7}-Netzwerk voraussetzen und nutzen.

Eine weitere Schwachstelle in \ac{GSM} ist die Möglichkeit, einen kryptografischen Schlüssel für mehrere Dienste hintereinander benutzen zu können. Ein Angreifer kann so auf Kosten und mit der Identität eines Opfers, Netzwerkdienste in Anspruch nehmen \citep{gsmmap:secrep-ger}. Das Vorgeben einer falschen Identität ist ebenfalls durch sogenanntes "`Call Spoofing"' möglich, also das Fälschen der Rufnummer des Anrufers. Der Dienst wird von mehreren Webseiten\footnote{\url{https://www.spoofcard.com/}} und Smartphone Apps\footnote{\url{http://calleridfaker.com/}} angeboten und ist in Deutschland nach \ac{TKG} §66k Rufnummernübermittlung verboten. Dabei wird der Anruf in der Regel über \ac{VoIP} Dienste geleitet, bei denen die Telefonnummer des Anrufers konfigurierbar ist. Auch Asterisk kann für Call-Spoofing genutzt werden. Die Anwendung erlaubt es dem Angreifer, seine Telefonnummer beliebig zu ändern, bevor ein Anruf an einen \ac{VoIP}-Dienst weitergeleitet wird.

\section{Vergleich mit dem neu entwickelten MitM-Angriff}

In dieser Masterarbeit wurde der erste \ac{MitM}-Angriff für \ac{GSM} entwickelt, der ohne Kenntnis des kryptografischen Schlüssels funktioniert. Der Angriff nutzt, wie der von \citet{bittau2006final} publizierte Angriff auf \ac{WEP}, den fehlenden Integritätsschutz aus, um verschlüsselte und bekannte Daten zu manipulieren. In Kombination mit vorhersehbarem Nachrichtenverkehr kann in \ac{GSM} so die Nachricht für den Anrufaufbau identifiziert und eine bekannte, angerufene Rufnummer geändert werden. Der Angreifer kann den Anruf so an ein Mobiltelefon in seinem Besitz umleiten. Verknüpft der Angreifer den bei seinem Mobiltelefon eingehenden Anruf mit einem Anruf zur ersetzten Nummer, kann er das Gespräch abhören und manipulieren -- ein \ac{MitM}-Angriff mit Zugriff auf die unverschlüsselten, zwischen den Opfern kommunizierten Sprachdaten.

Im Gegensatz zu den \ac{MitM}-Angriffen von \citet{barkan2003instant} und \citet{meyer2004man} beruht der vorgestellte Angriff nicht auf der Verwendung einer schwachen Verschlüsselung wie A5/1 oder A5/2. Da der Angriff keine Kenntnis des kryptografischen Schlüssels voraussetzt, funktioniert er mit allen spezifizierten Verschlüsselungen. Selbst wenn Netzwerk und \ac{MS} nur absolut sicher verschlüsselte Verbindungen zulassen (z.B. A5/4), kann der \ac{MitM}-Angriff durchgeführt werden. 

Mit der Spezifikation von A5/4 hat \ac{3GPP} ein Verschlüsselungsverfahren ohne die in \autoref{hdl:passive-attacks} gezeigen Schwächen spezifiziert. Der vorgestellte Angriff zeigt auf, dass alle Bemühungen, ein sicheres, auf Stromverschlüsselung basiertes  Verschlüsselungsverfahren für \ac{GSM} zu entwickeln, umsonst sind. Keine Stromverschlüsselung ist in der Lage, den fehlenden Integritätsschutz aufzuwiegen. Solange die Integrität einer Nachricht nicht durch einen sicheren Hashwert geschützt wird, kann ein Angreifer beliebige, ihm bekannte Daten ändern. Der Angriff nutzt diese Schwachstelle aus, um eine bekannte Telefonnummer im Anrufaufbau zu manipulieren. Es wäre jedoch genauso möglich, die oftmals bekannten Inhalte von Signalisierungs- oder anderer Nachrichten zu ändern. Angriffe, die auf solchen oder ähnlichen Manipulationen beruhen, müssen noch untersucht werden.

Mit dem \ac{MitM}-Angriff ist es mit der Manipulation von einigen wenigen Bits in einer einzigen Nachricht möglich, alle Sicherheits- und Netzwerkmechanismen für den weiteren Verlauf des Gesprächs zu umgehen. Ist der Angriff einmal aufgebaut, kann der Angreifer sich auf viel höherer Ebene mit dem Abhören oder Eingreifen in die Kommunikation beschäftigen - auf Ebene der Audiodaten des Gesprächs. Das Netzwerk und die beteiligten \acp{MS} erledigen Verschlüsselung, Kodierung und Frequency-Hopping, sowie den Handover der beiden Opfer und des Angreifers. Selbst ein Handover zu echten Basisstationen, die nicht mehr unter Kontrolle des Angreifers sind, ist möglich, was bedeutet, dass sich Angreifer und Opfer während eines laufenden \ac{MitM}-Angriffs beliebig bewegen können. Die geringe technische Komplexität des \ac{MitM}-Angriffs macht ihn als Dienstleistung für Angreifer mit wenig technischem Knowhow interessant. Vorstellbar wäre ein Szenario, in dem Überwachungsdienstleister X anbietet, ausgehende Anrufe von Opfer A an die Nummer von Opfer B über Angreifer Y als \ac{MitM} zu leiten. Der Dienstleister übernimmt dabei den \ac{MitM}-Angriff auf dem \ac{Um} und somit die technisch anspruchsvollere Umleitung des Anrufs und verkauft Angreifer Y den \ac{MitM}-Angriff auf die Sprachverbindung. Angreifer Y kann im einfachsten Fall "`einfach zuhören"'.

Für den \ac{MitM}-Angriff auf dem \ac{Um} ergibt sich eine neue Möglichkeit der Umsetzung, verglichen mit der herkömmlichen über eine falsche Basisstation. Wegen der geringen rechnerischen Komplexität der Manipulation der Setup-Nachricht im Vergleich zur Berechnung des kryptografischen Schlüssels könnte auf dem \ac{Um} ein rein auf physikalischer Ebene, in Echtzeit arbeitender \ac{MitM} zum Einsatz kommen. Das \ac{MitM}-Gerät würde in etwa einem \ac{GSM}-Repeater mit einer beim Empfang und Versand von Nachrichten vorgeschalteten Verarbeitungsroutine entsprechen, die die Bits im Datenstrom flippt. Hardware für \ac{GSM}-Repeater wird kostengünstig von verschiedenen Onlineshops angeboten \citep{teltarif:gsm-repeater}. Die Zeit für die Analyse und Manipulation der Daten (sieh \autoref{hdl:theroetical-attack}), könnte durch Ausnutzung der Timing Advance Funktionalität gewonnen werden (siehe \autoref{hdl:ta}). Mit Timing Advance ist der Gewinn von bis zu $233 \mu s$ für die Verarbeitung möglich.

Dass die mit \ac{UMTS} eingeführte, gegenseitige Authentifizierung keinen \ac{MitM}-Angriff verhindern kann, haben \citet{meyer2004man} gezeigt. Auch der vorgestellte \ac{MitM} Angriff funktioniert in \ac{GSM}-Netzwerken mit \ac{UMTS}-\ac{AKA}, da dieses in \ac{GSM} keinen Integritätsschutz bietet und Nachrichten unbemerkt von einem Angreifer manipuliert werden können.

Der vorgestellte \ac{MitM}-Angriff hat noch zwei Nachteile, die durch Gegenmaßnahmen behoben werden können. Erstens sehen die Opfer die Rufnummer des Angreifers, an den das Telefonat umgeleitet wird. Mit Call-Spoofing kann der Angreifer seine eigene Identität verschleiern und den Opfern die erwartete Identität vorgegeben. Zweitens ergibt sich eine größere Signallaufzeit für die übertragenen Daten, da diese von Opfer 1 zuerst zum Angreifer und dann zu Opfer 2 geschickt werden. Der Angreifer kann den Effekt möglichst gering halten, indem er sich in der Nähe von einem der beiden Opfer aufhält. Die zusätzliche Latenz durch den höheren \ac{TA}-Wert wegen einem \ac{MitM} auf der physikalischen Ebene fällt nicht ins Gewicht.