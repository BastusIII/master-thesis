%%%%%%%%%%%%%%%%%%%%%%%%%%%%%%%%%%%%%%%%%%%%%%%%%%%%%%%%%%%
%%% Ziel
%%%%%%%%%%%%%%%%%%%%%%%%%%%%%%%%%%%%%%%%%%%%%%%%%%%%%%%%%%%
\chapter{Ziel} \label{hdl:ziel}
Ziel dieser Arbeit ist es, einen \ac{MitM}-Angriff auf eine verschlüsselte Sprachverbindung im \ac{GSM}-Netz umzusetzen. Der Angriff soll den fehlenden Integritätsschutz der Kommunikation zwischen Netzteilnehmer und Netzwerk ausnutzen, um ein ausgehendes Telefonat an eine vom Angreifer bestimmte Telefonnummer umzuleiten. Die Verschlüsselung soll nicht gebrochen werden, der kryptografische Schlüssel wird also als unbekannt vorausgesetzt.

Der Angriff soll sowohl theoretisch entwickelt, als auch praktisch umgesetzt werden. Im theoretischen Teil soll die Möglichkeit der Identifizierung der für den Anrufaufbau zuständigen Nachricht im Nachrichtenfluss gezeigt und die Machbarkeit der Manipulation dieser Nachricht, ohne Kenntnis des kryptografischen Schlüssels, mathematisch nachgewiesen werden. Im praktischen Teil der Arbeit soll der Angriff innerhalb einer Testumgebung, anhand eines vom Opfer ausgehenden Anrufs, praktisch durchgeführt werden. Die Implementierung des Angriffs und dafür nötiger Anwendungen soll das Open-Source-Projekt Osmocom nutzen. Für den Aufbau der Testumgebung sollen die Projekte osmoBTS (die \ac{BTS}) und osmocomBB (die \ac{MS}) über eine virtuelle Funkschnittstelle verbunden werden. Die virtuelle Funkschnittstelle soll die Übertragung der Nachrichten über \ac{UDP}/\ac{IP}, statt einem Funksignal ermöglichen. Für die virtuelle Funkschnittstelle soll eine \ac{MitM}-Anwendung implementiert werden, die Zugriff auf die Kommunikation zwischen \ac{MS} und \ac{BTS} hat. Der ausgehende Anruf des Opfers soll vom \ac{MitM} an eine vom Angreifer bestimmte Rufnummer umgeleitet werden. Die Weiterleitung der Audiodaten, vom Mobilfunktelefon des Angreifers an die ersetzte Rufnummer, ist nicht Teil dieser Arbeit, ebenso wie die Umsetzung des \acp{MitM}-Angriffs auf der realen Funkschnittstelle.

Das übergreifende Ziel ist es, den Bedarf an Integritätsschutz generell und speziell in \ac{GSM} zu verdeutlichen. Es soll gezeigt werden, dass die fehlende Integrität die Verschlüsselung nutzlos macht, wenn ein Angreifer Teile der verschlüsselten Signalisierungsnachrichten kennt. 