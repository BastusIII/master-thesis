%%%%%%%%%%%%%%%%%%%%%%%%%%%%%%%%%%%%%%%%%%%%%%%%%%%%%%%%%%%
%%% Ergebnis
%%%%%%%%%%%%%%%%%%%%%%%%%%%%%%%%%%%%%%%%%%%%%%%%%%%%%%%%%%%
\chapter{Das Ergebnis der Arbeit} \label{hdl:ergebnis}

Das Ergebnis dieser Arbeit ist die Entwicklung und erfolgreiche Durchführung eines praktischen \ac{MitM}-Angriffs auf \ac{GSM}. Der Vorteil gegenüber anderen aktuellen Angriffen liegt in seiner Unabhängigkeit vom verwendeten Verschlüsselungsverfahren. Da der Angriff ohne Kenntnis des kryptografischen Schlüssels auskommt, ist das Brechen der Verschlüsselung nicht notwendig. Das ermöglicht seine Ausführung in vollständig mit sicheren Algorithmen wie A5/3 oder A5/4 geschützten Verbindungen.

Der Angriff wurde erfolgreich in einer Testumgebung durchgeführt und seine praktische Machbarkeit dadurch verifiziert. Für die Testumgebung wurden die Osmocom Projekte osmoBTS und osmocomBB, die die \ac{BTS} und die \ac{MS} realisieren, modifiziert und um eine virtuelle \ac{Um}-Schnittstelle erweitert. Für die Durchführung des Angriffs wurde eine \ac{MS} über eine \ac{MitM}-Implementierung  im virtuellen \ac{Um} mit der \ac{BTS} verbunden. Die \ac{MS} tätigte anschließend einen Anruf an eine bekannte Rufnummer. Die \ac{MitM}-Implementierung konnte mit der eingebauten \ac{IMSI}-Catcher Funktionalität das Opfer anhand seiner \ac{IMSI} erfolgreich identifizieren und den ausgehenden Anruf diesem zuordnen. Über die vor dem Anrufaufbau ausgetauschten, unverschlüsselten Signalisierungsnachrichten zwischen \ac{MS} und \ac{BTS} konnte die \ac{MitM}-Implementierung die für den Anrufaufbau zuständige, verschlüsselte Setup-Nachricht im Nachrichtenfluss erfolgreich identifizieren. Der letzte Schritt des durchgeführten Angriffs bestand in der Manipulation der angerufenen Nummer in der Setup-Nachricht. Ohne Kenntnis des für die Generierung des Schlüsselstroms verwendeten kryptografischen Schlüssels war es möglich, die Telefonnummer in der verschlüsselten und kodierten Setup-Nachricht zu lokalisieren und bekannte Ziffern zu manipulieren. Das Ergebnis des durchgeführten Angriffs war die erfolgreiche Umleitung eines vom Opfer ausgehenden Anrufs an eine vom Angreifer bestimmte Rufnummer.

Da auf der virtuellen Funkschnittstelle \ac{GSMTAP}-Nachrichten über \ac{UDP}-Multicast-Sockets kommuniziert werden, konnte der gesamte durchgeführte Angriff von Wireshark aufgezeichnet werden (siehe \autoref{lst:mitm_attack_wireshark}).

Die Implementierung der virtuellen \ac{Um}-Schnittstelle wurde des Weiteren auf der OsmocDevCon2017, der Entwicklerkonferenz des Osmocom Projekts, vorgeführt und positiv aufgenommen. Aktuell liegen die Implementierungen noch in eigenen Branches des osmoBTS und osmocomBB Projekts, sollen aber als nächster Schritt in den Masterbranch integriert werden. Des Weiteren gibt es Pläne, das virtuelle \ac{Um} als Grundlage für schnelle und hardwareunabhängige Tests im relativ neuen Projekt osmoTester zu integrieren. 