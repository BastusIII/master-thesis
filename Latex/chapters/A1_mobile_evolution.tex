\chapter{Entwicklung des Mobilfunks} \label{hdl:einleitung_entwicklung}

Der Grundstein für \ac{GSM} wurde bereits 1982 auf der \ac{CEPT} mit der Gründung der "`Groupe Spécial Mobile"' gelegt. Deren Aufgabe war es, einen einheitlichen Standard für die europäischen Mobilfunknetze zu entwickeln. Die Arbeit an der Standardisierung wurde ab 1988 nach und nach von der \ac{ETSI} übernommen und die "`Groupe Spécial Mobile"' ein Komitee der \ac{ETSI}. 1990 wurden die Spezifikationen des \ac{GSM}-900-Standards eingefroren und als Basis für den Aufbau der Infrastruktur des Mobilfunknetzes und die Herstellung von Mobiltelefonen verwendet. Ab 1991 begannen die Netzbetreiber, auf Messen für das neue Mobilfunknetz zu werben, der kommerzielle Start des \ac{GSM}-900 Netzes dauerte aber noch bis 1992. In Deutschland traten als erste Betreiber die Deutsche Telekom (D1 Netz) und Mannesmann Mobilfunk (D2 Netz) auf.
2000 wurden alle \ac{GSM}-Standards von \ac{ETSI} nach \ac{3GPP} überführt. Aktuell werden sie dort von der \ac{GERAN} Arbeitsgruppe weiterentwickelt und gepflegt.
Seit der Einführung von \ac{GSM} wurden die Standards mehrfach überarbeitet, erweitert und an neue Technologien angepasst. Im Folgenden wird die Geschichte der Entwicklung des Mobilfunks nach \citet{handyflatrate24:umts-gprs-lte-und-co} und \citet{3gpp:gsm-history} kurz aufgeführt. 

\begin{description}
\item[Ab Ende 50er -- Diverse analoge Mobilfunknetze (1G)]
Die analogen Systeme für Mobilfunk vor \ac{GSM} waren nicht einheitlich und länderübergreifend standardisiert, teuer und unhandlich. In Deutschland waren das A-, B- und C-Netz der Deutschen Bundespost verbreitet.
\item[1992 -- \acl{GSM} (\acs{GSM} - 2G)]
Seit 1992 ersetzt \ac{GSM} als europäischer Standard die vorherigen Mobilfunknetze. Neben Übertragung von Sprachdaten bietet \ac{GSM} noch die Möglichkeit, Kurzmitteilungen über \ac{SMS} zu versenden und stellt Roaming Dienste für länderübergreifende Telefonie zur Verfügung. Der Standard wird als 2. Generation (2G) bezeichnet und ist der digitale Nachfolger der veralteten analogen Netze. Mit ca. 700 \ac{GSM}-Mobilfunknetzen in 200 Ländern ist \ac{GSM} noch immer eines der verbreitetsten Netze weltweit.
\item[2000 -- \ac{HSCSD}]
Die Unterscheidung mehrerer Mobiltelefone auf der Luftschnittstelle (siehe \autoref{hdl:einfuehrung-gsm_schnittstellen_protokolle-um_interface}) zur Basisstation funktioniert über \ac{TDMA}. Eine Verbindung zwischen zwei Mobiltelefonen belegt dabei einen Kanal, was einem \ac{TDMA} Timeslot entspricht. Pro Kanal können 14,4 kbit/s übertragen werden. \ac{HSCSD} ermöglicht es einem Mobiltelefon bis zu 4 Kanäle für die Datenübertragung zu belegen. Dadurch erhöht sich die Auslastung der \ac{BTS} bei wenigen aktiven Netzteilnehmern und die maximale Transferrate einer Verbindung steigt auf 57,6 kbit/s.
\item[2000 -- \acl{GPRS} (\acs{GPRS} - 2.5G)]
\ac{GSM} verwendet Leitungsvermittlung, um eine Verbindung zwischen zwei Netzteilnehmern aufzubauen. Über den reservierten physikalischen Kanal können für die Dauer der Verbindung Daten übertragen werden. Für Telefonie ist das sinnvoll, da während der Dauer eines Gesprächs kontinuierlicher Datenverkehr zwischen den Teilnehmern besteht. \ac{GPRS} erweitert \ac{GSM} um paketorientierte Datenübertragung. Statt eine dauerhafte Verbindung zu reservieren, werden zu sendende Daten in Pakete aufgeteilt, die nur bei Bedarf verschickt werden. Damit können mehr Nutzern gleichzeitig Datendienste zur Verfügung gestellt werden und die Kapazität des Netzes wird besser ausgenutzt.
\ac{GPRS} ermöglicht theoretisch Datenraten von 110kBit/s. Um \ac{GPRS} in das Netzwerk einzubinden, muss das \ac{NSS} um \ac{SGSN}, \ac{GGSN} und das \ac{BSS} um eine \ac{PCU} erweitert werden (siehe \fullref{hdl:einfuehrung-gsm_architektur}). Für die \ac{BTS} müssen zudem neue Kanäle für die Paketdatenübertragung konfiguriert werden. Da die Infrastruktur des \ac{GSM}-\ac{BSS} weiter verwendet wird, spricht man von der 2.5ten Generation.
\item[2005/2006 -- \ac{EDGE}]
 \ac{EDGE} führt eine neue Modulationsart ein, welche erhöhte Datenraten für \ac{HSCSD} und \ac{GPRS} erlaubt.  \ac{EDGE} erhöht zudem die Stabilität der Datenübertragung. Mit \ac{EDGE} können bei Belegung von 4 Timeslots bis zu 220kBit/s im Downlink und 110kBit/s im Uplink erreicht werden.
\item[2004 --  \acl{UMTS} (\acs{UMTS} - 3G)]
Der  \ac{UMTS}"=Standard wird als 3. Generation bezeichnet, da er nicht länger auf der bestehenden \ac{BSS} Infrastruktur aufbaut, sondern mit \ac{UTRAN} eine neue Infrastruktur benötigt. Das Core Network wird unter dem Namen \ac{GERAN} größtenteils weiterverwendet. Der Standard schließt Sicherheitslücken in \ac{GSM} und erhöht Stabilität und Datenrate von Verbindungen. Eine wichtige Änderung war die Überarbeitung des Authentifizierungsverfahrens. Wo das \ac{GSM}-\ac{AKA} noch keine Möglichkeit bietet, die Authentizität der \ac{BTS} zu überprüfen \citepauthor{3gpp:03.20}, wird mit dem \ac{UMTS} \ac{AKA} eine gegenseitige Authentifizierung eingeführt \citepauthor{3gpp:33.102}. Unterstützen \ac{AuC} des Netzanbieters und Mobiltelefon das neue Authentifizierungsverfahren, so kann es auch in nur 2G-fähigen \acp{BSS} verwendet werden -- für den \ac{GSM}-Standard ein deutliches Sicherheitsupgrade.
Auf Architektur und Spezifikationen des  \ac{UMTS}-Standards wird in dieser Arbeit nicht näher eingegangen.
\item[2006 -- \acl{HSPA} (\acs{HSPA} - 3.5G)]
Ein Standard zur weiteren Erhöhung der  \ac{UMTS} Datenraten durch Optimierung der Datenübertragung.
\item[2010 -- \acl{LTE} (\acs{LTE} - 4G)]
Der aktuellste Mobilfunkstandard mit zukunftssicheren Datenraten bis theoretisch 300Mbit/s, Sicherheit, Stabilität und geringer Latenz. Das Mobilfunknetz der 4. Generation setzt komplett auf paketorientierte Übertragung und ist nicht kompatibel zu  \ac{UMTS} Hardware. Die Netzinfrastruktur muss für flächendeckende \ac{LTE} Unterstützung also ausgebaut werden. Mit der \ac{VoLTE} Erweiterung ist auch Sprachübertragung möglich, für die davor auf ältere Standards zurückgegriffen wurde. Vodafone machte 2015 den Anfang, inzwischen unterstützen laut \citetauthor{lte-anbieter:volte} alle deutschen Anbieter \ac{VoLTE}. Die Weiterentwicklung des Standards, \ac{LTE}-Advanced, ermöglicht Datenraten bis 1000Mbit/s. Aktuell bieten in Deutschland sowohl Vodafone als auch Telekom eine sehr gute \ac{LTE}-Abdeckung \citep{ltemap:ltemap}\citep{opensignal}. Auf Architektur und Spezifikationen des \ac{LTE}-Standards wird in dieser Arbeit nicht näher eingegangen.
\item [2014 -- 5G]
Die Entwicklung eines 5G Standards wurde 2014 von Huawei auf der eigens dafür einberufenen \href{http://www.5gsummit.org/}{5G@Europe Summit} initiiert. Führende europäische Netzbetreiber forschen seitdem an dem neuen Standard, der mit bis zu 10 GBit/s, niedrigeren Latenzzeiten und Energieverbrauch sowie einer deutlich größeren Kapazität das 4G Netz ablösen soll. Die Standardisierung wurde im Juni 2016 mit Release 15 veröffentlicht und soll im September 2018 abgeschlossen sein \citepauthor{3gpp:releases}. Erste Testnetzwerke mit Datenraten von bis zu 3-4 GBit/s wurden auf Messen bereits installiert, für 2020 sind erste Installationen für Endnutzer vorausgesagt \citep{lte-anbieter:5g}.
\end{description}